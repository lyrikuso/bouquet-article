% A4 段組なし 用紙は横置きにする
\documentclass[uplatex,a4paper,oneside,landscape]{jsarticle}

% フッターにノンブル
\pagestyle{plain}

% 見出しレベル
\setcounter{secnumdepth}{0}

\usepackage[sourcehan-otc]{pxchfon}
\usepackage{lltjp-geometry}

\makeatletter
% 縦組にする
\AtBeginDocument{\tate\adjustbaseline}

% \textheight と \textwidth を逆転
\setlength{\@tempdima}{\textheight}
\setlength{\textheight}{\textwidth}
\setlength{\textwidth}{\@tempdima}

% ヘッダの左右を逆転
\let\@temp\@evenhead
\let\@evenhead\@oddhead
\let\@oddhead\@temp
\fullwidth\textheight

% \maketitle の出力様式を微修正
\def\@maketitle{%
  \newpage\null
  \vskip 2em
  \begin{center}%
    {\huge \gtfamily \@title \par}%
    \vskip 1.5em
    {\large
      \lineskip .5em
        \@author
      \par}%
  \end{center}%
  \par\vskip 1.5em
}

\let\plainifnotempty\relax

% 40字30行 geometryに任せるので30行にならないことがある
\usepackage[truedimen,textwidth=40zw,lines=30,centering]{geometry}

\makeatother

\title{「スイミング・スクール」を読む}
\author{小峰さちこ}
\date{}

\begin{document}
\maketitle

\section{フラットな語り⼝とその事情}

鈴⽊ちはね特集をやるよということでお声がけをいただいたので、鈴⽊の「スイミング・スクール」を題材に評をやってみようと思う。といっても、特定の短歌作品を緻密に読み解いて何ごとかを述べるような評論というのは、率直に⾔って、私の得意とするところでない。ついでに正直に告⽩すると、そもそも私は短歌をどこかに投稿するという営みにあまり関⼼がなく、賞を獲ったという連作を読みこむみたいな経験も今回がはじめてのことだった。だから、勝⼿がよくわからない。そこで以下では、ひとまず⼆三川練による「第⼆回笹井宏之賞を読んだ話」という⽂章の論旨を紹介しながら、私なりに「スイミング・スクール」を読み解くためのひとつの補助線としてみたい。

この⽂章の冒頭、⼆三川は「短歌は既に多様性の時代に突⼊している」としたうえで、短歌作品の作中主体として想定される〈私〉が従来より多⾯的な、作中を通して必ずしも⼀貫したキャラクタへと還元しえないような⾃⼰像を志向しつつあるらしいことを指摘する。そのうえで、読者にそのような⾃⼰像を読み取らせるための⼿⽴てとして、第⼆回笹井宏之賞の受賞作群は次の三つの技法を駆使することで、読者から通俗的な物語として共感されることを拒否しようとしているという。

\begin{itemize}
\item
  ⾔葉を喩ではなく論理で繋ぐ
\item
  ⼼象より具象の描写にコストを払う
\item
  具体物に過剰な意味を持たせない
\end{itemize}

とりわけ、本来ならば「⾔葉の論理的関係を断ち切り互いを喩的関係に置くことで詩的抒情を発⽣させる」ものであるという句切れへの期待を逆⼿にとり、その期待を裏切ることによって、かえって⾔葉を「淡く論理的(≒フラット)」に配置しているとする⾔及は切れ味鋭く、⼆三川の評者としての確かな⼒量を感じさせる。そのように⾔葉どうしを句切れをまたぎながらあえてフラットに接続することによって⽣じる効果をさして、⼆三川は「反神話的感懐」と呼ぶ。

\begin{quote}
「神話」とは、読者が作品に期待するある種の定型化した感動、⽂脈のことを⾔う。「物語」と呼ばれることの⽅が多いだろうが、本記事では出来事の流れやその羅列を包括するものを「物語」と呼び、その⼀カテゴリとして「神話」を位置づけようと思う。(中略)つまり「反神話」とは共同体の拒否であり、僕が先⽇のブログで語ったような「個⼈」を獲得するための⼀⼿段である。
\end{quote}

ここで⼆三川が述べている「個⼈」とは、共同体の想像⼒によって補完されることなく⽴ち現れる〈私〉のことであり、したがって、理想的には社会⽂化的な個別の⽂脈に依存せずに「ジャンルを横断する」ことが可能な⾃⼰像であるという。素朴な前提として、そのときどきに⽂脈を注⼊し、キャラクタライズされながら⽴ち現れる⾃⼰の表象というのは、必ずしもいまここにおける〈私〉の意に沿うような⾃⼰像とはかぎらない。そのような必ずしも意に沿わない⾃⼰像をいわば無理やりあてがわれ、〈⾃分〉として引き受けさせられることは、いまここにおける〈私〉にとって忌避されるべき暴⼒にほかならない。それを避けるべくあえてフラットな語り⼝が採⽤されているのだという指摘は⼀⾒もっともらしいもので、確かに説得⼒がある。

しかし、本論において私は、フラットな語り⼝がなされているという点では⼆三川と同様の⾒⽅を踏襲しつつも、「スイミング・スクール」でそのような語り⼝が実現している経緯については、⼆三川とは異なるストーリーを描くことを試みたい。あらかじめ⼿短に述べてしまうと、おそらく⼆三川は、⾃⾝の関⼼に引きつけるかたちで、誰もが〈自分〉として語りだそうとする自己像をもっているということを前提としたうえで、それが疎外されることを避けるための議論を展開している。そこにおける語り⼝の選択とは、あくまで主体的になされたものとして、語り⼿の表現の⽬的を叶えるためのレトリックのひとつと位置づけられている。⼀⽅で私は、けっして鈴⽊の詠み⼿としての⼒量を侮るわけではないけれども、「スイミング・スクール」におけるフラットな語り⼝は、むしろ主体がそのようにしか⾃⼰を語りだすための⾔葉をもたなかったがために、結果としてそのようにならざるをえなかった性質のものだと考える。

こう考えるヒントのひとつは、⼆三川も紹介している、鈴⽊が⾃⾝の受賞に寄せたコメントの放つ奇妙な違和感のなかにある。

\begin{quote}
鈴⽊の受賞の⾔葉には「短歌を続けるということはつねに何かを再定義し続けるということで、(中略)僕は⾃分が(他ならぬ)(どうしようもない)⾃分でしかありえないということを引き受けるために、短歌定型の⼒をここまで借りてきたのではないか」とある。この⼀連には⼈物感とでも呼ぶべきものはあるがいわゆる⼈物像のようなものは無い。つまり「私」のキャラクタライズはほとんど⾏われていない。「ほとんど」というのは、連作が神話を拒否しているのは確かだが、事象の羅列という意味での物語は健在しているからである。そのなかで描かれている⼀貫した「私」は確かにおり、それは僕には⼀様に思えたからである。果たしてこれで「⾃分」を描けたのかは疑問であるが、このように反神話的感懐を⼗分に駆使した連作が受賞作に選ばれたことは喜ばしいことだろう。
\end{quote}

ここで⼆三川が覚えている違和感はもっともなもので、私たちはふつう、語りの主体がただ⼀貫しているというだけでは、そうして語りだされた⾃⼰像を〈⾃分〉とは呼ばない。私たちが語りだす⾃⼰像とは、何の脈絡もないものとしてあるがままに記述されるものではない。それらは、語り⼿のそのときどきの視点や意図のもとで、いわば物語的に編成されて提⽰される。私たちは、いまここにおける〈私〉の物語にとって意味がある任意の要素だけを選択・強調することによって、ただの〈私〉ならざる〈⾃分〉という表象をアドホックに構築して語りだしているのである。

そうした⾃⼰像としての〈⾃分〉は、あるいはひとつの連作という単位のなかで、あるいは⼀⾸の独⽴した短歌という単位のなかで、あるいはもっとミニマルな場合では句や節のような単位のなかで、ともかくも⼀連のまとまりのなかへ差し出されることを志向した表象として⽴ち現れるものである。けれども、「スイミング・スクール」に垣間⾒える⾃⼰像は、そうして素直に⼀連のまとまりのなかへ差し出されることを断念せざるをえなかった結果であるかのように、どこか底知れないフラットさを湛えているように映る。では、鈴⽊がそのような語り⼝を採らざるをえなかった事情とはいったいどのようなものなのだろうか。

\section{⼼象のない世界}

さしあたって強調しておきたいのは、⼆三川の読みと私の読みとが前提としているものの根本的な違いだ。おそらく⼆三川は、⾃⾝の関⼼に引きつけるかたちで、誰もが〈自分〉として語りだそうとする自己像をもっているということを前提としたうえで、それが疎外されることを避けるための議論を展開している。だが私は、どのような詠み⼿の思惑のうちにもつねに〈自分〉として語りだされるべき⾃⼰像があるはずだという点について、これを揺るぎない前提とは認めない。奇妙な⾔い⽅だが、鈴⽊の意識はむしろ、どうすれば⾃⼰を語らずに⾃⼰なるものを引き受けていくことが可能かという⼀点を模索しているように⾒える。つまり、鈴⽊の「⾃分が(他ならぬ)(どうしようもない)⾃分でしかありえないということを引き受けるために、短歌定型の⼒をここまで借りてきた」という⾔葉の背後にあるのは、〈自分〉として語りだされるべき⾃⼰像という、本当はあってしかるべき語りへの動機づけの不在であり、定型の⼒を借りてきたというのはすなわち、その動機づけの不在を定型という〈私〉の外なる語らせる⼒によって代替してきたということなのだと解釈したい。

このような地点からスタートして、物語的に語りだされる〈⾃分〉以前の〈私〉の語りを読み解こうとする試みは、昨今となってはさほど⽬新しいものでもないはずだ。少なくとも、私が以下の議論の直接の参考としている「乖離する私――中村⽂則――」という⽂芸評論の初出は『群像』の⼆〇〇六年六⽉号である。

さて、「乖離する私」の筆者である⽥中弥⽣は、ちょうどそのころに芥川賞受賞者として名を連ねた⾦原ひとみ、綿⽮りさ、中村⽂則といった作家の共通項として「主⼈公の感情を本⼈の⽬の端に、『私』が⾒ていないもの(=背景)として書く」という特徴を指摘している。

それはたとえば、綿⽮が『蹴りたい背中』の主⼈公に「さびしさは鳴る」というあの印象的な書き出しを語らせたことに象徴される、感情というものに対するある種の認知的な癖のようなものだ。私たち⼈類はもともと、ホックシールドが「感情労働」という⽤語をもって鮮やかに暴いてみせたように、〈私〉の感情が疎外されることをあえて引き受けることによって、はじめて社会参加を許されるという側⾯がある。そういった営みのなかへ新たに参⼊していく若年者の⼼理的葛藤というのは、数多の⽂学作品がすでに扱ってきたテーマでもあるだろう。
ところで⽥中は、そのような葛藤(と、それに対する共感)が成⽴するのは、「社会的な感情乖離技術の習得以前に、より根源的な感情認識回路と、それが絶対になくならないという確信がある⼈間においてのみである」と指摘する。つまり、いまここにおける〈私〉の感情が、〈私〉の⽴ち位置から⾒える背景の⼀部として認識されるようなタイプの⼈間にあっては、感情はそもそも〈私〉の内⾯に属することではないため、それらが取り⽴てて意識されうるチャンスが失われているのである。

\begin{quote}
なんとなく知らない⾞⾒ていたら持ち主にすごく睨まれていた

不動産屋の前に⽴ち⽌まって⾒ていると不動産屋が中から⾒てくる
\end{quote}

⼆三川はこの⼆⾸を例に挙げて、「『睨まれていた』ことのバツの悪さや『中から⾒てくる』ことの居⼼地の悪さは感じさせる」ものの、あくまで出来事の描写に終始している(というよりも、そのようなやり⽅で描写に徹する態度が連作全体を通底している)と指摘するのだが、それこそ、⼆三川⾃⾝があるいはそうかと勘づいている読者の⽴場からの「神話的期待」に他ならないだろう。この連作中の主体は、少なくとも表現として表れてくるレベルでは、感情を⾃分の内⾯に帰属させることはないように思える。もしも意識されないながらに感情が機能することがあったとしても、それは典型的には次のように、むしろ外的な環境の描写として表現されることだろう。

\begin{quote}
スロープと階段があってスロープのほうを下ればよろこびがある
\end{quote}

この「下れば・よろこびが・ある」という独特な表現は、何かの狙いのもとに認識を捻った結果としてこうなったのではない。⼆三川は受賞作群の全体を⾒渡して、⼼象よりも具象の描写により多くのコストが払われていることを指摘していたが、これは必ずしも「スイミング・スクール」の正鵠を真正⾯から射抜くものではなかった。そこでは『蹴りたい背中』の世界と同様に、具象だけが存在を許されているのであって、語られるべき⼼象などはじめから存在しないものなのである。

\section{透明な⾝体}

⼈々のあいだに⾒られるこうした認知の癖の異なりについて、⽥中は論の運びの都合上、「世代」というタームを⽤いながら相当の分量を割いて説明している。しかし、それは厳密な世代論を⽬指したものというより、今も昔も、⼈々のあいだにはそういった異なりがときに⾃覚されないまま横たわっていることがある、というくらいの話だろう。さて、⽥中のいう、感情が絶対になくならないという確信があった世代(第⼀世代)では、主体が演じる⾏為とその背後にあるべき動機づけとは⾃ずから結びついているものだった。⼀⽅で、それに後続する世代(第⼆世代)においては、⾏為の動機づけというのは〈私〉の内⾯に属するものではない。それらはむしろ〈私〉の外部にある環境の⼀部として視界の裡に据えられるものだ。

第⼀世代的な素朴な世界観のなかでは、⾏為は感情という内なる動機づけを基盤として駆動していた。そこでは、ある⾏為について、なぜそうするのかという動機づけが⾒失われることはない。⼀⽅、第⼆世代では感情は描写の対象ではあるものの、それはなぜ他ならぬこの〈私〉がそうせざるをえなかったのかという内的根拠の説明にはなりえない。そこでは、真なる内的根拠の説明はつねに先延ばしにされている。ホックシールドによって発⾒された「感情労働」という概念は、今⽇の第⼆世代的な世界観のなかにあって、個性のキャラ化(キャラクタライゼーション)として、ほとんどあらゆる⽣活場⾯のなかに敷衍して適⽤されるようになった。キャラクタとは、第⼆世代にとって内なる動機づけを代替するための装置であり、ある状況を刺激として受け取ると、その状況に応じて主体が演じるべき適切な⾏為を返す〈メソッド〉の束のようなものだ。第⼆世代の〈私〉たちの⾝体は、この擬‐感情としてのキャラクタを上⼿く演じるために、透明な神籬のような姿をしている(ここで「神籬」という読者の⽿慣れないだろう語を⽤いたのは、もちろん、⼆三川がこうした共同体の想像⼒を「神話」に擬えたことを念頭に置いている)。

第⼆世代的な世界を⽣きる「スイミング・スクール」の主体は、なぜそうせざるをえなかったのかという説明を引き受けることに対してリスク回避的な態度を貫いている。その姿勢は端的には、⼆三川も指摘している「淡く論理的」な語り⼝のその徹底ぶりのなかに表れているのだが、それと同時に〈私〉をして⾏為の内的根拠の説明へと向かわしめる社会の〈まなざし〉のようなものをひどく気にしているらしい様⼦からもうかがわれる。

\begin{quote}
⼤⾬のニュースを⾒てる 意味もなく必要以上に部屋を暗くして

年⾦を払ってないと来る電話が払ってからはもう来なくなる

交番に誰もいないのをいいことに交番の前を通り過ぎた
\end{quote}

このようになんだか意味のよくわからない⾏為の「意味のわからなさ」という点について先回りして意識しているのは、その説明を要請された場合のリスクを本能的に予感しているからだ。あるいはまた先ほど取り上げた「睨まれていた」とか「中から⾒てくる」といった作品などは、その〈まなざし〉をより直接的に描いたものだろう。⼆三川は「具体物に過剰な意味を持たせない」という⾔い⽅をしていたが、「スイミング・スクール」は〈私〉にとっての意味を仔細に描きこまないくせに、それを問うてくるだろう他者の〈まなざし〉についてはむしろ頻りに意識を割いている。

ところで、ここでこの主体にとって本当に切実なことというのは〈私〉をそのように⾏為せしめる内的根拠がこの〈私〉にもよくわからないという状態そのものではない。

\begin{quote}
どんぐりを⾷べた記憶があるけれどどうやって⾷べたかわからない

いま何を聞かれても⼝ごもるだろう いつ 何を どうして どこまで
\end{quote}

このどうしようもないわからなさというのは主体にとってかえって所与の感覚なのであって、それに由来する当惑が本当に形を得て姿を現すまでにはまだ⼀刻の猶予がある。それでは、モラトリアムはいつ終わるのか。それはこの〈私〉の透明な⾝体が他者の〈まなざし〉の前に差し出される、具体的な対⼈場⾯をおいて他にない。

\begin{quote}
くら寿司で暮らすし⼤丈夫 親に笑いながら殺すなよと⾔われても
\end{quote}

「くら寿司で暮らすし⼤丈夫」というこの何の説明にもなっていない軽⼝は、したがって、明らかな虚勢と⾒るべきだろう。この主体には、確かな実感を伴いながら語りうるような内的根拠の説明というものが、ほとんど絶望的なほどに、⾒つけられないのである。空間を遥か底まで⾒通せるということは、かといって、空間を⾒透かせるということに直結しない。透明な⾝体を有するこの主体にとって、真なる内的根拠の説明とは、〈私〉という透明な⽔で満たされたプールの⽔中から任意のイオンを探し集めるような⾏いによく似ている。あるいは知識としてそういうものがあるということは理解できていたとしても、それは具象としてそこに⾒きれてはいない以上、彼にとっては存在しないも同然のものである。それを析出するためにはある技術が必要だった。それは必ずしも特別な技ではないはずなのだが、しかし、その技術というのは、彼の⽣きる第⼆世代の世界にあってはすでに失われた技術なのである。

\section{短歌定型という明かり}

ここまでくるとやや深読みが過ぎる⾃覚もあるのだが、⾒渡すかぎり〈私〉という透明な⽔ばかりが占めて⾒える空間に、いわば「慣れ親しむ」ための⾝体的技術を教わる場であるという点において、「スイミング・スクール」というタイトルはきわめて象徴的だと思う。

\begin{quote}
スイミングスクール通わされていた夏の道路の明るさのこと
\end{quote}

連作のタイトルはこの⼀⾸から採られている。それというのも、⾃分の意志でそうしたという主体的な選択の記憶ではない。それは(あるいは作中主体の親だろうか)他者からの求めに応じるかたちで「通わされていた」場所だった。けれども、それが「夏の道路の明るさ」の記憶だというのは、「この部屋のここだけ蛍光灯」とか「眠れない夜には散歩する」といった暗さのほうが⽬に付きがちな連作全体の空気のなかで、特別な意味を帯びているように⾒える。

\begin{quote}
⼯事中の⽩い壁まぶしく光る 義務教育の⼦どもが通る
\end{quote}

連作中でこのように描かれる〈⼦ども〉という存在は、なんというか、作中主体とは時間的に距離を隔てた、まだ夏の⽇差しがさすような明るい地点に⽴っている。彼らは、この連作の主体のありえたかもしれない反実仮想の姿である(この連作における「明るさ」はただの明るさではなく、夏というモチーフの内部に配置された「明るさ」であるという点で意味を帯びている。蛇⾜っぽい想像だが、品川の⼿前で⽬覚めて東京の夏の終わりを⾒届けたはずの⼀⾸の後、あいだでパンツを四つ並べて⼲してはいるものの、すでに⼗⽉へと向かっているはずの同じ電⾞の⾞内に、主体の反実仮想としての〈⼦ども〉が配置されているのは、そういう対照的な⽴ち位置を際⽴たせる演出だろう)。

『蹴りたい背中』は、主⼈公たちが若かったがゆえに、いわばまだ引き返すことが可能な地点で起こる物語だった。事実として綿⽮が描いたのは、ありていに⾔ってしまえば、これから第⼆世代になろうとする若者が⾃分たちの透明な⾝体の内部に感情回路を再発⾒する希望の物語である。しかしながら、スイミングスクールに通った夏をとうに乗り越してきてしまった⼤⼈である〈私〉たちには、引き返すという選択肢はもう残されていない。⽥中の評論が⾦原・綿⽮よりもむしろ重点的に取り上げた中村⽂則の初期三部作(『銃』、『遮光』、『⼟の中の⼦供』)は、そのような⾝体をそなえた〈私〉たちが⾃らの内部に感情回路を再発⾒するよりも先に、とうとう死に戻りを繰り返すしかなくなる、殺伐としたロールプレイングゲームのような物語だった。それは⽥中に⾔わせれば、物語としてどうしようもない「進まなさ」を感じさせるものである。もちろんそれは停⽌ではなく遅さであり、実際、中村は『⼟の中の⼦供』に⾄ってようやくわずかながらの希望を書き残すことに成功している。とはいえ、そのためには〈私〉たちの残機を⼆機費やさなければならなかったというのもまた事実であり、その「進まなさ」は、実際問題として現実世界に⽣きる⽣⾝の〈私〉たちにとっては、やはり致命的な遅さなのである。

\begin{quote}
乖離の直前にあるものが回路を発⾒して引き返す話を書く綿⽮りさは速い。避難は速くなければ意味がないのだ。だが、⼀旦乖離してしまった⼈間が感情に帰ることの不可能さを速さは再現できない。
\end{quote}

しかし、幸か不幸か、鈴⽊は⼩説家ではなく、歌⼈だった。彼の⼿許には、ときに「⼀⼈称の⽂学」などと仰々しく呼ばれることさえある「短歌」というきわめて強⼒なメディアがあった。それは〈自分〉として語りだされるべき⾃⼰像など持ち出さなくとも、この〈私〉に何ごとかを語らせしめ、他ならない〈私〉による語りとしてパッケージ化してしまう、ほとんど呪術的といってもよいほどに強⼒な照明装置である。

\begin{quote}
核の傘 あるいは喩から想起するほんとうに⾒えている核の傘
\end{quote}

「核の傘」というのは実際、現実世界のどこにも実体をもたないという意味で、いわば「神話」の世界にしか存在しないものだ。しかし、短歌定型における⽐喩というレトリックは、確かな実感を伴いながらそれを描写する〈私〉を強⼒に措定する共同体の作⽤によって、たとえ実際には存在しないものであっても、その虚影の内側から逆向きの光を照射し、具象としての「ほんとう」の姿を照らし出すことを可能にしている。では、もしそれがレトリックではなく、詩型そのものにそなわった働きなのだとしたらどうだろうか。そんな洞察に希望を託し、⽐喩を論理形式へと置換して、⾒渡すかぎり「淡く論理的」でしかないフラットな語りの内側から、どこにもないはずの内的根拠の姿を炙り出そうとした実験の場こそが「スイミング・スクール」という連作だった。

では、その試みは成功を収めただろうか。率直に⾔って、それは私にはよくわからない。本論の冒頭でも⾔い訳したことだが、特定の短歌作品を緻密に読み解いて何ごとかを述べるような評論というのは、そもそも私の得意とするところでない。けれど、少し安⼼したような気はする。「スイミング・スクール」のように、意味を過剰に描きこまないことで、かえってその語り⼝を意味ありげなものに⾒せているような短歌作品というのは、ときどき⽬にするものだ。しかし、⼆三川の読みがそうだったように、その寡黙さをまともに受け取った者が、そこに共同体の想像⼒の拒絶を⾒出すという構図は、私たちのメディア・リテラシーというものに対する不信感の根深さを思わせる。それはちょうど『⼟の中の⼦供』の芥川賞選評のなかで村上⿓が「読者」への絶望を吐露した様⼦に重なるものだ。

\begin{quote}
(被虐待者は)「微妙で切実で制御不能なストレスと不具合を併せ持って」おり、それは「本⼈も理解できていない場合も多いから」、「そういった⼈を主⼈公にして⼀⼈称で⼩説を書くのは、読者との距離感を意図的に崩した緻密で実験的な⽂体が必要になる」
(中略)
もちろん、村上⿓は書き⼿である。これは読み⼿としての意⾒ではなく、彼⾃⾝が持つ、読み⼿への絶望の吐露と⾒るべきだろう。「誠実な⼩説家は、そんなことは不可能だと思わなければならない」という⾔葉の妙に切実な響きは、「やっても誰も読まない」という彼⾃⾝の対読者不信を感じさせる。
\end{quote}

けれど、少なくとも本論の描いたストーリーのなかでは、歌⼈は「読者」による「神話的期待」のなかにこそ〈私〉たちを照らしうる光を⾒出した。共同体の想像⼒というのは結局、私たちのメディア・リテラシーの謂いである。リテラシーは確かに個⼈に属する技術だが、それは、だからといって〈私〉以外に頼れる読者がいないということとイコールではない。〈私〉に読み解けないことなら、別の誰かに読み解いてもらえばいいのだ。本論の描くストーリーが鈴⽊の実際の思惑と重なるものかはわからないけれど、そのような読みを託すにも⼗分な余⽩をそなえた「スイミング・スクール」の語り⼝は、私にはとてもしたたかなものに思えた。


\begin{thebibliography}{99}
\item 鈴⽊ちはね(2020)「スイミング・スクール」(『短歌ムック ねむらない樹
  vol.4』書肆侃侃房 pp.8-11)
\item ⽥中弥⽣(2006)「乖離する私――中村⽂則――」(『群像 61 (6) 』 講談社
  pp.150-167)
\item ⼆三川練(2020-01)「『個⼈』を獲得するために〜寺⼭修司が遺した課題〜」
  (http://23rivers.blog.fc2.com/blog-entry-40.html)
\item ⼆三川練(2020-02)「第⼆回笹井宏之賞を読んだ話」
  (http://23rivers.blog.fc2.com/blog-entry-41.html)
\end{thebibliography}

\end{document}
